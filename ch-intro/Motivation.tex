\section{Motivation and Goals}
\label{sec:introduction:motivation}

This work is motivated by the observation that both airfoil ice
accretion and airplane cargo hold fires are fundamentally uncertain
problems, which makes certification for safe flight challenging. Both
of these safety concerns involve a significant amount of input process
uncertainty, and the statistical effects of this uncertainty on
important safety/performance metrics has not heretofore been
thoroughly studied.

The goal of this work is to (1) show how input stochastic processes
can be efficiently modeled, and (2) investigate the statistical
effects of this uncertainty on important output quantities of
interest. In the icing problem, we will investigate how aerodynamic
performance metrics (e.g., lift) are affected statistically by
uncertainty in the ice shape or in underlying physical conditions. In
the cargo hold fire problem, we will show how boundary condition
uncertainty (e.g., position/heat flux of the fire source) can affect
measures of fire detection (e.g., ceiling temperature distribution).


