
\chapter{Introduction\label{ch:intro}}

In an aggregate sense, aviation is one of the safetest means of
transportation that exists today. According to statistics provided by
the National Transportation Safety Board (NTSB), of the $34,678$
transportation fatalities registered in 2013, only $443$ were related
to aviation, compared to $32,719$ highway-related fatalities, $891$
rail-related fatalities, and $615$ marine-related
fatalities~\cite{ntsbStats}. Ensuring the continuation and improvement
of this tradition of safe airplane flight is the responsibility of the
Federal Aviation Administration (FAA), which proposes legislation
aimed at establishing certification rules to establish the
airworthiness of airplanes.

Two longstanding phenomena that pose issues for aircraft safety are
aircraft icing and cargo hold fires. Both of these problems are
explicitly addressed in Title 14 of the Code of Federal Regulations
(14 CFR), part 25 (Airworthiness Standards: Transport Category
Airplanes). The importance of ice accretion is underscored by the
recent 2014 amendment to the CFR, which enacts stricter certification
rules for flight in supercooled large droplet (SLD) icing
conditions~\cite{fedregisterSLD}. The importance of cargo hold fires
is demonstrated by the tight requirements for fire detection systems,
which must alert the flight crew within 60 seconds of
ignition~\cite{CFR}.

Given the recognized safety issues presented by both icing and fires,
there has been much research conducted in these arenas. As we will see
in this thesis, the literature encompasses work that addresses informational
deficiencies across a wide range of topics, including computation,
experimental observations, fundamental physical understanding,
sensitivity analyses/optimizations, etc.

However, both ice accretion and cargo hold fires are processes which
are, in practice, subject to a wide range of uncertainty. The sources
of these uncertainties are manifold, but what is important is that the
existing literature lacks a thorough treatment of this
topic. Stated plainly, this is the central objective of this thesis --
to address the topic of uncertainty in ice accretion and cargo hold
fire problems. The work done in this arena falls into one of a few
categories: describing/modeling sources of uncertainty, developing
and/or implementing methods for quantifying uncertainty, and analyzing
the relative contributions to variance in a particular output quantity
of interest. For completeness, we will address all three of these
categories in this thesis.

\section{Motivation and Goals}
\label{sec:introduction:motivation}

This work is motivated by the observation that both airfoil ice
accretion and airplane cargo hold fires are fundamentally uncertain
problems, which makes certification for safe flight challenging. Both
of these safety concerns involve a significant amount of input process
uncertainty, and the statistical effects of this uncertainty on
important safety/performance metrics has not heretofore been
thoroughly studied.

The goal of this work is to (1) show how input stochastic processes
can be efficiently modeled, and (2) investigate the statistical
effects of this uncertainty on important output quantities of
interest. In the icing problem, we will investigate how aerodynamic
performance metrics (e.g., lift) are affected statistically by
uncertainty in the ice shape or in underlying physical conditions. In
the cargo hold fire problem, we will show how boundary condition
uncertainty (e.g., position/heat flux of the fire source) can affect
measures of fire detection (e.g., ceiling temperature distribution).



\section{Approach}
\label{sec:introduction:approach}

Here is some text for the approach introduction section.

\section{Contributions}
\label{sec:introduction:contributions}

Here is some text for the contributions introduction section.

